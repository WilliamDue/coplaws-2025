\documentclass[aspectratio=169]{beamer}

\usepackage[utf8]{inputenc}
\usepackage[english]{babel}
\usepackage[T1]{fontenc}
\usepackage{lmodern}
\usepackage{bbm}
\usepackage{listings}
\usepackage{multicol}
\usepackage{csquotes}
\usepackage{hyperref}
\usepackage{indentfirst}
\usepackage{amsthm}
\usepackage{amsfonts}
\usepackage{amsmath}
\usepackage{amssymb}
\usepackage{mathtools}
\usepackage{tikz}
\usepackage{xcolor}
\usepackage{array}
\usepackage{bookmark}
\usepackage{booktabs}

\newcommand{\id}[1]{\ensuremath{\mathit{#1}}}
\newcommand{\kw}[1]{\ensuremath{\mathtt{#1}}}
\newcommand{\Let}{\kw{let}}
\newcommand{\In}{\kw{in}}
\newcommand{\If}{\kw{if}}
\newcommand{\Then}{\kw{then}}
\newcommand{\Else}{\kw{else}}
\newcommand{\Filter}{\kw{filter}}
\newcommand{\Or}{\kw{or}}
\newcommand{\Groupby}{\kw{groupby}}
\newcommand{\Partition}{\kw{partition}}
\newcommand{\Scatter}{\kw{scatter}}
\newcommand{\Hist}{\kw{hist}}
\newcommand{\Map}{\kw{map}}
\newcommand{\Sum}{\kw{sum}}
\newcommand{\Mod}{\kw{mod}}
\newcommand{\Fst}{\kw{fst}}
\newcommand{\Snd}{\kw{snd}}
\newcommand{\Presum}{\kw{presum}}
\newcommand{\True}{\kw{true}}
\newcommand{\False}{\kw{false}}
\newcommand{\Sort}{\kw{sort}}
\newcommand{\Segor}{\kw{segor}}
\newcommand{\Hash}{\kw{hash}}
\newcommand{\Random}{\kw{random}}
\newcommand{\Iota}{\kw{iota}}
\newcommand{\Unzip}{\kw{unzip}}
\newcommand{\Unit}{\mathbf{unit}}
\newcommand{\Int}{\mathbf{int}}
\newcommand{\Bool}{\mathbf{bool}}
\newcommand{\Def}{\kw{def}}
\newcommand{\While}{\kw{while}}
\newcommand{\Do}{\kw{do}}
\newcommand{\Rep}{\kw{rep}}
\newcommand{\Zip}{\kw{zip}}

\usetikzlibrary{automata, positioning, arrows}

\graphicspath{ {./figures/} }

\definecolor{palepurple}{rgb}{0.7647,0.69411,0.88235}
\definecolor{palepurple}{rgb}{0.7647,0.69411,0.88235}
\definecolor{midviolet}{rgb}{0.6,0.4,0.8}

\newcolumntype{M}[2]{>{\centering\arraybackslash}m{#1}<{\rule{0pt}{#2}}}
\newcommand{\darkpalepurple}{palepurple!70!black}
\definecolor{paleyellow}{rgb}{1.0,1.0,0.7725}
\setbeamercolor{background canvas}{bg=paleyellow}

\usetheme{default}
\usecolortheme{wolverine}
\setbeamercolor*{palette primary}{bg=palepurple}
\setbeamercolor*{palette quaternary}{bg=palepurple}
\setbeamercolor{structure}{fg=palepurple}
\setbeamertemplate{navigation symbols}{\insertframenumber/\inserttotalframenumber}
\setbeamertemplate{blocks}[rounded][shadow]
\setbeamercolor{frametitle}{bg=midviolet,fg=black}
\setbeamercolor{block title}{bg=structure, fg=black}
\setbeamercolor{block body}{bg=structure, fg=black}
\setbeamercolor{block title alerted}{bg=structure, fg=black}
\setbeamercolor{block body alerted}{bg=structure, fg=black}
\setbeamercolor{block title example}{bg=structure, fg=black}
\setbeamercolor{block body example}{bg=structure, fg=black}
\setbeamercolor{frametitle}{bg=palepurple,fg=black}


\setbeamertemplate{itemize item}{\scriptsize$\blacksquare$}
\setbeamertemplate{itemize subitem}{\scriptsize$\blacksquare$}
\setbeamertemplate{itemize subsubitem}{\scriptsize$\blacksquare$}

\title[Hash Maps]{Functional Hash Maps in a Data Parallel Language}
\author{\textbf{William Henrich Due} \inst{1} \and Martin Elsman \inst{1} \and Troels Henriksen \inst{1}}
\institute[shortinst]{\inst{1} Department of Computer Science, University of Copenhagen}
\date{August 22nd, 2025}

\begin{document}

\begin{frame}
  \begin{center}
    \titlepage
    \vfill
    \usebeamerfont{institute} Contact: \url{widu@di.ku.dk}
  \end{center}
\end{frame}

\begin{frame}\frametitle{Open Adressing Example}
  \begin{columns}
  \begin{column}{.48\textwidth}
  \hfill
  \begin{itemize}
    \item Keys $k_0, k_1 \in K$.
    \item Hash function $h: K \to \{0, 1, 2, 3\}$.
    \item $h(k_0) = h(k_1) = 1$.
  \end{itemize}
  \hfill  
  \end{column}
  \hfill
  \begin{column}{.5\textwidth}
    $$
    \begin{array}{|M{1cm}{0.4cm}|M{1cm}{0.4cm}|M{1cm}{0.4cm}|M{1cm}{0.4cm}|}
      \hline
      & \only<2,4->{$k_0$} \only<3>{$k_0, \color{red}{k_1}$} & \only<4->{$k_1$} & \\ \hline
    \end{array}
    $$
  \end{column}
  \end{columns}
\end{frame}

\begin{frame}\frametitle{Core Ideas}
  \begin{columns}
  \begin{column}{.38\textwidth}
  \hfill
  \begin{itemize}
    \item Concurrency.
    \item Collision resolution.
  \end{itemize}
  \hfill  
  \end{column}
  \hfill
  \begin{column}{.6\textwidth}
       $$
    \begin{array}{|M{1cm}{0.4cm}|M{1cm}{0.4cm}|M{1cm}{0.4cm}|M{1cm}{0.4cm}|}
      \hline
      & $k_0$ & $k_1$ & \\ \hline
    \end{array}
    $$
    \begin{center}
      or
    \end{center}
    $$
    \begin{array}{|M{1cm}{0.4cm}|M{1cm}{0.4cm}|M{1cm}{0.4cm}|M{1cm}{0.4cm}|}
      \hline
      & $k_1$ & $k_0$ & \\ \hline
    \end{array}
    $$
  \end{column}
  \end{columns}
\end{frame}

\begin{frame}\frametitle{Hash Maps in Functional Data Parallel Languages}
  \begin{columns}
  \begin{column}{.38\textwidth}
  \hfill
  \begin{itemize}
    \item Avoid collisions.
    \item Bulk operations.
    \item<2-> Fredman-Komlós-Szemerédi (FKS) construction. 
  \end{itemize}
  \hfill  
  \end{column}
  \hfill
  \begin{column}{.6\textwidth}
    \begin{align*}
      \Map & : (\alpha \to \beta) \to [n]\alpha \to [n]\beta \\
      \mathtt{from\_array} & : [n](\alpha, \beta) \to \mathbf{hashmap} ~ \alpha ~ \beta
    \end{align*} 
  \end{column}
  \end{columns}
\end{frame}

\begin{frame}\frametitle{Perfect Hashing with FKS}
  \begin{center}
    \begin{tikzpicture}
    % Vertical positions
    \def\yTop{3.0}
    \def\yMid{0.0}
    \def\yarrays{-2.0}

    % Horizontal positions for arrays
    \def\xA{0}          % First 4-cell array (left)
    \def\xB{6}          % Single cell array (middle)
    \def\xC{7.8}        % Second 4-cell array (right)

    % Wider cells for hash buckets
    \def\cellW{1.7}
    \def\cellH{1.2}

    % Center everything properly
    \def\totalWidth{12}  % Total diagram width
    \def\topArrayWidth{6} % 5 cells * 1.2 = 6
    \def\midArrayWidth{8.5} % 5 cells * 1.7 = 8.5
    \def\xTop{3}      % Center top array: (12-6)/2 = 3
    \def\xMid{1.75}   % Center middle array: (12-8.5)/2 = 1.75
    
    \foreach \i in {0,...,4} {
      \draw (\xTop+\i*1.2,\yTop) rectangle (\xTop+\i*1.2+1.2,\yTop+1);
      \node at (\xTop+\i*1.2+0.6,\yTop+0.5) {$k_{\i}$};
    }
    \onslide<2->{
      % Middle array: 5 bigger cells for "bins", only some filled with sets
      \foreach \i in {0,...,4} {
        \draw[thick, densely dashed] (\xMid+\i*\cellW,\yMid) rectangle (\xMid+\i*\cellW+\cellW,\yMid+\cellH);
      }
    }
    \node[font=\footnotesize] at (\xTop-0.5,\yTop+0.5) {Keys}
    \onslide<2->{
      % Only fill h_1, h_2, h_4
      \node[text=red] at (\xMid+1*\cellW+0.5*\cellW,\yMid+0.6) {$k_{0},\,k_{3}$};
      \node[text=blue] at (\xMid+2*\cellW+0.5*\cellW,\yMid+0.6) {$k_{1}$};
      \node[text=green!70!black] at (\xMid+4*\cellW+0.5*\cellW,\yMid+0.6) {$k_{2},\,k_{4}$};
    }

    \onslide<3->{
      % Bottom arrays (unchanged)
      % First array of 4 simple boxes (left)
      \foreach \j in {0,...,3} {
        \draw (\xA+\j*1.2,\yarrays) rectangle (\xA+\j*1.2+1.2,\yarrays-1);
      }
      \node[text=red] at (\xA+0.6,\yarrays-0.5) {$k_3$}; % first cell
      \node[text=red] at (\xA+3.0,\yarrays-0.5) {$k_0$}; % fourth cell
      
      % Single box array (middle)
      \draw (\xB,\yarrays) rectangle (\xB+1.2,\yarrays-1);
      \node[text=blue] at (\xB+0.6,\yarrays-0.5) {$k_1$}; % middle box
      
      % Second array of 4 simple boxes (right)
      \foreach \j in {0,...,3} {
        \draw (\xC+\j*1.2,\yarrays) rectangle (\xC+\j*1.2+1.2,\yarrays-1);
      }
      \node[text=green!70!black] at (\xC+1.8,\yarrays-0.5) {$k_4$}; % second cell
      \node[text=green!70!black] at (\xC+3.0,\yarrays-0.5) {$k_2$}; % fourth cell
    }
    
    \onslide<2->{
      \node[font=\footnotesize] at (\xMid-0.8,\yMid+0.6) {Collisions};
      % Arrows from topmost k_i to bins - First show all keys going to h
      % All keys point to a central h point - center it between top and middle
      \pgfmathsetmacro{\hPointX}{\xTop+2.5*1.2}
      \pgfmathsetmacro{\hPointY}{\yTop-0.9}
      
      % Lines from each key to the central h point
      \draw[->, thick] (\xTop+0.6,\yTop) -- (\hPointX,\hPointY);
      \draw[->, thick] (\xTop+3*1.2+0.6,\yTop) -- (\hPointX,\hPointY);
      \draw[->, thick] (\xTop+1*1.2+0.6,\yTop) -- (\hPointX,\hPointY);
      \draw[->, thick] (\xTop+2*1.2+0.6,\yTop) -- (\hPointX,\hPointY);
      \draw[->, thick] (\xTop+4*1.2+0.6,\yTop) -- (\hPointX,\hPointY);
      
      % Arrows from h to each bin
      \draw[->, thick, red] (\hPointX,\hPointY) -- (\xMid+1*\cellW+0.5*\cellW,\yMid+\cellH);
      \draw[->, thick, blue] (\hPointX,\hPointY) -- (\xMid+2*\cellW+0.5*\cellW,\yMid+\cellH);
      \draw[->, thick, green!70!black] (\hPointX,\hPointY) -- (\xMid+4*\cellW+0.5*\cellW,\yMid+\cellH);
      
      % Label the hash function h at the central point
      \node[font=\footnotesize, circle, draw, fill=palepurple] at (\hPointX,\hPointY) {$h$};
    }

    \onslide<3->{
      % Arrows: h_1, h_2, h_4 to bottom arrays (h_1 arrow RED, h_2 arrow BLUE, h_4 arrow GREEN)
      \draw[->, thick, red] (\xMid+1*\cellW+0.5*\cellW,\yMid) -- (\xA+2.4,\yarrays); % h1 to center of left 4-cell array
      \node[font=\footnotesize] at (\xMid+1*\cellW-0.9,\yMid-1.0) {$h_0$};
      
      \draw[->, thick, blue] (\xMid+2*\cellW+0.5*\cellW,\yMid) -- (\xB+0.6,\yarrays); % h2 to center of 1-cell array
      \node[font=\footnotesize] at (\xMid+2*\cellW+0.5*\cellW-0.3,\yMid-1.0) {$h_1$};
      
      \draw[->, thick, green!70!black] (\xMid+4*\cellW+0.5*\cellW,\yMid) -- (\xC+2.4,\yarrays); % h4 to center of right 4-cell array
      \node[font=\footnotesize] at (\xMid+4*\cellW+0.2*\cellW+0.3,\yMid-1.0) {$h_2$};
      \node[font=\footnotesize] at (\xA-0.8,\yarrays-0.5) {Subarrays};
    }
    \end{tikzpicture}
  \end{center}
\end{frame}

\begin{frame}\frametitle{Flattening the finding of collision-free hash functions.}
  \begin{center}
    \resizebox{.8\columnwidth}{!}{
      \begin{tikzpicture}
        % Vertical positions
        \def\yTop{2.0}
        \def\yMid{0.0}
        \def\yarrays{-1.0}

        % Horizontal positions for arrays
        \def\xA{0}          % First 4-cell array (left)
        \def\xB{6}          % Single cell array (middle)
        \def\xC{7.8}        % Second 4-cell array (right)

        % Wider cells for hash buckets
        \def\cellW{1.7}
        \def\cellH{1.2}

        % Center everything properly
        \def\totalWidth{12}  % Total diagram width
        \def\topArrayWidth{6} % 5 cells * 1.2 = 6
        \def\midArrayWidth{8.5} % 5 cells * 1.7 = 8.5
        \def\xTop{3}      % Center top array: (12-6)/2 = 3
        \def\xMid{1.75}   % Center middle array: (12-8.5)/2 = 1.75
        
        \foreach \i in {0,...,4} {
          \draw (\xTop+\i*1.2,\yTop) rectangle (\xTop+\i*1.2+1.2,\yTop+1);
          \node at (\xTop+\i*1.2+0.6,\yTop+0.5) {$k_{\i}$};
        }
        % Middle array: 5 bigger cells for "bins", only some filled with sets
        \foreach \i in {0,...,4} {
          \draw[thick, densely dashed] (\xMid+\i*\cellW,\yMid) rectangle (\xMid+\i*\cellW+\cellW,\yMid+\cellH);
        }
        \node[font=\footnotesize] at (\xTop-0.5,\yTop+0.5) {Keys}
        % Only fill h_1, h_2, h_4
        \node[text=red] at (\xMid+1*\cellW+0.5*\cellW,\yMid+0.6) {$k_{0},\,k_{3}$};
        \node[text=blue] at (\xMid+2*\cellW+0.5*\cellW,\yMid+0.6) {$k_{1}$};
        \node[text=green!70!black] at (\xMid+4*\cellW+0.5*\cellW,\yMid+0.6) {$k_{2},\,k_{4}$};

        % Bottom arrays (unchanged)
        % First array of 4 simple boxes (left)
        \foreach \j in {0,...,3} {
          \draw (\xA+\j*1.2,\yarrays) rectangle (\xA+\j*1.2+1.2,\yarrays-1);
        }
        \node[text=red] at (\xA+0.6,\yarrays-0.5) {$k_3$}; % first cell
        \node[text=red] at (\xA+3.0,\yarrays-0.5) {$k_0$}; % fourth cell
        
        % Single box array (middle)
        \draw (\xB,\yarrays) rectangle (\xB+1.2,\yarrays-1);
        \node[text=blue] at (\xB+0.6,\yarrays-0.5) {$k_1$}; % middle box
        
        % Second array of 4 simple boxes (right)
        \foreach \j in {0,...,3} {
          \draw (\xC+\j*1.2,\yarrays) rectangle (\xC+\j*1.2+1.2,\yarrays-1);
        }
        \node[text=green!70!black] at (\xC+1.8,\yarrays-0.5) {$k_4$}; % second cell
        \node[text=green!70!black] at (\xC+3.0,\yarrays-0.5) {$k_2$}; % fourth cell
      
        \node[font=\footnotesize] at (\xMid-0.8,\yMid+0.6) {Collisions};
        % Arrows from topmost k_i to bins - First show all keys going to h
        % All keys point to a central h point - center it between top and middle
        \pgfmathsetmacro{\hPointX}{\xTop+2.5*1.2}
        \pgfmathsetmacro{\hPointY}{\yTop-0.4}
        
        % Lines from each key to the central h point
        \draw[->, thick] (\xTop+0.6,\yTop) -- (\hPointX,\hPointY);
        \draw[->, thick] (\xTop+3*1.2+0.6,\yTop) -- (\hPointX,\hPointY);
        \draw[->, thick] (\xTop+1*1.2+0.6,\yTop) -- (\hPointX,\hPointY);
        \draw[->, thick] (\xTop+2*1.2+0.6,\yTop) -- (\hPointX,\hPointY);
        \draw[->, thick] (\xTop+4*1.2+0.6,\yTop) -- (\hPointX,\hPointY);
        
        % Arrows from h to each bin
        \draw[->, thick, red] (\hPointX,\hPointY) -- (\xMid+1*\cellW+0.5*\cellW,\yMid+\cellH);
        \draw[->, thick, blue] (\hPointX,\hPointY) -- (\xMid+2*\cellW+0.5*\cellW,\yMid+\cellH);
        \draw[->, thick, green!70!black] (\hPointX,\hPointY) -- (\xMid+4*\cellW+0.5*\cellW,\yMid+\cellH);
        
        % Label the hash function h at the central point
        \node[font=\footnotesize, circle, draw, fill=palepurple] at (\hPointX,\hPointY) {};

        % Arrows: h_1, h_2, h_4 to bottom arrays (h_1 arrow RED, h_2 arrow BLUE, h_4 arrow GREEN)
        \draw[->, thick, red] (\xMid+1*\cellW+0.5*\cellW,\yMid) -- (\xA+2.4,\yarrays); % h1 to center of left 4-cell array
        
        \draw[->, thick, blue] (\xMid+2*\cellW+0.5*\cellW,\yMid) -- (\xB+0.6,\yarrays); % h2 to center of 1-cell array
        \draw[->, thick, green!70!black] (\xMid+4*\cellW+0.5*\cellW,\yMid) -- (\xC+2.4,\yarrays); % h4 to center of right 4-cell array

        \node[font=\footnotesize] at (\xA-0.8,\yarrays-0.5) {Subarrays};
      \end{tikzpicture}
    }
  \end{center}
  \begin{columns}
  \begin{column}{.43\textwidth}
  \hfill
  \begin{align*}
    \onslide<1->{& \Map \: \lambda subarrays \to \\
    & \quad \While \: h_i \text{ leads to collisions} \: \Do \\
    & \quad \quad \text{Pick a random hash function } h_i}
  \end{align*}
  \hfill  
  \end{column}
  \hfill
  \begin{column}{.1\textwidth}
  \hfill
  \begin{align*}
     \onslide<2->{\mapsto}
  \end{align*}
  \hfill
  \end{column}
  \begin{column}{.43\textwidth}
  \hfill
  \begin{align*}
    \onslide<2->{
    & \While \: \text{any collisions in subarrays} \: \Do \\
    & \quad \Map \: \lambda keys \to \text{pick new hash functions}}
  \end{align*}
  \hfill  
  \end{column}
  \end{columns}
\end{frame}

\begin{frame}\frametitle{Benchmarks}
  \begin{center}
    \begin{tabular}{l|rrr}
      & \multicolumn{3}{c}{\textbf{64-bit integer keys} ($n=10^{7}$)} \\
      & \textit{Construction} & \textit{Lookup} \\\midrule
      Futhark (hash maps) & 18.3 & 3.3  \\
      Futhark (binary search) & 40.9 & 6.2 \\
      Futhark (Eytzinger) & 42.3 & 4.3 \\
      cuCollections & $2.7$ & $1.1$ \\
    \end{tabular}
    \vspace{0.5cm}
    
    All times in milliseconds measured on an A100 GPU.
  \end{center}
\end{frame}

\begin{frame}\frametitle{The End}
  \begin{center}
    \textbf{Towards Efficient Hash Maps in Functional Array Languages}\\
    \vspace{0.25cm}
    \url{https://arxiv.org/abs/2508.11443}

    \vspace{1cm}
    \textbf{Code} \\
    \vspace{0.25cm}
    \url{https://github.com/diku-dk/containers}

    \vspace{0.25cm}
    \url{https://github.com/diku-dk/futhark-hashmap-experiments}
  \end{center}
\end{frame}

\end{document}
